\documentclass[10pt]{article}

\usepackage{cite,url}
\usepackage{checker}

\title{An Extensible Proof Checker of the Simply Typed $\lambda$-Calculus}
\author{Yuting Wang}


\begin{document}
\maketitle

\section{The simply typed $\lambda$-calculus}

The following is the syntax of the simply typed $\lambda$-calculus.
%
\begin{tabbing}
\qquad\=$\tau$\quad\=$:=$\quad\=\kill
\>$\tau$  \>$:=$  \>$a \sep \tau_1 \to \tau_2$\\
\>$t$     \>$:=$  \>$c \sep x \sep \tabs{x}{\tau} t \sep t_1\app t_2$
\end{tabbing}
%
where $a$ denotes constant types, $c$ denotes constants and $x$
denotes variables.

Let $\Sigma := c_1:\tau_1, \ldots, c_n:\tau_n$ denote the signature
and $\Phi =: x_1:\tau_1,\ldots,x_n:\tau_n$ denote the type context.
The typing rules derives judgments of the form
$\typeof{\Sigma}{\Phi}{t}{\tau}$ using the rules in
Figure~\ref{fig:stlc-typing-rules}:

\begin{figure}[ht!]
\fbox{$\typeof{\Sigma}{\Phi}{t}{\tau}$}
\begin{gather*}  
\infer[\mbox{\typconst}]{
  \typeof{\Sigma}{\Phi}{c}{\tau}
}{
  c:\tau \in \Sigma
}
\qquad
\infer[\mbox{\typvar}]{
  \typeof{\Sigma}{\Phi}{x}{\tau}
}{
  x:\tau \in \Phi
}
\\
\infer[\mbox{\typabs}]{
  \typeof{\Sigma}{\Phi}{\tabs{x}{\tau_1}{t}}{\tau_1 \to \tau_2}
}{
  x:\_ \not\in\Phi 
  &
  \typeof{\Sigma}{\Phi,x:\tau_1}{t}{\tau_2}
}
\qquad
\infer[\mbox{\typapp}]{
  \typeof{\Sigma}{\Phi}{t_1\app t_2}{\tau_2}
}{
  \typeof{\Sigma}{\Phi}{t_1}{\tau_1 \to \tau_2}
  &
  \typeof{\Sigma}{\Phi}{t_2}{\tau_1}
}
\end{gather*}  

\caption{Typing rules for the simply typed $\lambda$-calculus}
\label{fig:stlc-typing-rules}
\end{figure}


\section{\STLCE: an extension of the \STLC with explicit conversion rules}
%
We extend the \STLC with a proposition that describes the equality of
terms modulo $\beta$-conversion. To simplify the discussion we assume
that $\alpha$-conversion is implicitly applied when necessary. For
this, we introduce a type constant $\prop$ for classifying
propositions and the proposition $t_1 = t_2$ to denote equality
between $t_1$ and $t_2$. We also introduce explicit proof terms
denoted by $\pi$ to characterize the provability of propositions. We
call this extension \STLCE. The syntax of \STLCE is as follows, where the
syntax newly introduced w.r.t \STLC is marked red:
%
\begin{tabbing}
\qquad\=$\tau$\quad\=$:=$\quad\=\kill
\>$\tau$  \>$:=$  \>$\emphred{\prop} \sep a \sep \tau_1 \to \tau_2$\\
\>$t$     \>$:=$  \>$c \sep x \sep \tabs{x}{\tau} t \sep t_1\app t_2 \sep t_1 = t_2$\\
\>$\emphred{\pi}$   \>$:=$  \>$\emphred{c_\pi \sep \refl{t} \sep \trans{\pi_1}{\pi_2} \sep \symm{\pi}}$\\
\>\>\>$\emphred{\sep \congtabs{x}{\tau}{\pi} \sep \congappl{\pi} \sep \congappr{\pi} 
       \sep \congeql{\pi} \sep \congeqr{\pi}}$\\
\>\>\>$\emphred{\sep \tbetaconv{x}{\tau}{t_1}{t_2} \sep \conv{\pi_1}{\pi_2}}$
\end{tabbing}
%
The proof terms $\refl{t}$, $\trans{\pi_1}{\pi_2}$ and $\symm{\pi}$
capture the reflexivity, transitivity and symmetry properties of the
equality propositions; $\congtabs{x}{\tau}{\pi}$, $\congappl{\pi}$,
$\congappr{\pi}$, $\congeql{\pi}$ and $\congeqr{\pi}$ capture the
congruence of equality; $\tbetaconv{x}{\tau}{t_1}{t_2}$ captures
$\beta$-conversion; and finally, $\conv{\pi_1}{\pi_2}$ denotes the
explicit conversion between equivalent propositions.

The typing rules of \STLCE are described in Figure~\ref{fig:stlce-typing-rules}.
Note that we have extended the signature $\Sigma$ with assignments of
terms to proof-level constants. The rule $\typpconst$ derives the
types of proof-level constants from these assignments. The rest of the new
proof rules for typing proof terms are self-explanatory.

\begin{figure}[ht!]
\fbox{$\typeof{\Sigma}{\Phi}{t}{\tau}$}
\begin{gather*}  
\infer[\mbox{\typconst}]{
  \typeof{\Sigma}{\Phi}{c}{\tau}
}{
  c:\tau \in \Sigma
}
\qquad
\infer[\mbox{\typvar}]{
  \typeof{\Sigma}{\Phi}{x}{\tau}
}{
  x:\tau \in \Phi
}
\\
\infer[\mbox{\typabs}]{
  \typeof{\Sigma}{\Phi}{\tabs{x}{\tau_1}{t}}{\tau_1 \to \tau_2}
}{
  x:\_ \not\in\Phi 
  &
  \typeof{\Sigma}{\Phi,x:\tau_1}{t}{\tau_2}
}
\qquad
\infer[\mbox{\typapp}]{
  \typeof{\Sigma}{\Phi}{t_1\app t_2}{\tau_2}
}{
  \typeof{\Sigma}{\Phi}{t_1}{\tau_1 \to \tau_2}
  &
  \typeof{\Sigma}{\Phi}{t_2}{\tau_1}
}
\\
\infer[\mbox{\typteq}]{
  \typeof{\Sigma}{\Phi}{t_1 = t_2}{\prop}
}{
  \typeof{\Sigma}{\Phi}{t_1}{\tau}
  &
  \typeof{\Sigma}{\Phi}{t_2}{\tau}
}
\end{gather*}
\\
\fbox{$\typeof{\Sigma}{\Phi}{\pi}{t}$}
\\
\begin{gather*}
\infer[\mbox{\typpconst}]{
  \typeof{\Sigma}{\Phi}{c_\pi}{t}
}{
  c_\pi:t \in \Sigma
}
\qquad
\infer[\mbox{\typrefl}]{
  \typeof{\Sigma}{\Phi}{\refl t}{t = t}
}{
  \typeof{\Sigma}{\Phi}{t}{\tau}
}
\qquad
\infer[\mbox{\typtrans}]{
  \typeof{\Sigma}{\Phi}{\trans{\pi_1}{\pi_2}}{t_1 = t_3}
}{
  \typeof{\Sigma}{\Phi}{\pi_1}{t_1 = t_2}
  &
  \typeof{\Sigma}{\Phi}{\pi_2}{t_2 = t_3}
}
\\
\infer[\mbox{\typsymm}]{
  \typeof{\Sigma}{\Phi}{\symm{\pi}}{t_1 = t_2}
}{
  \typeof{\Sigma}{\Phi}{\pi}{t_2 = t_1}
}
\qquad
\infer[\mbox{\typcongabs}]{
  \typeof{\Sigma}{\Phi}{\congtabs{x}{\tau}{\pi}}{\tabs{x}{\tau}{t_1} = \tabs{x}{\tau}{t_2}}
}{
  \typeof{\Sigma}{\Phi,x:\tau}{\pi}{t_1 = t_2}
}
\\
\infer[\mbox{\typcongappl}]{
  \typeof{\Sigma}{\Phi}{\congappl{\pi}}{t_1\app t_2 = t_1'\app t_2}
}{
  \typeof{\Sigma}{\Phi}{\pi_1}{t_1 = t_1'}
}
\qquad
\infer[\mbox{\typcongappr}]{
  \typeof{\Sigma}{\Phi}{\congappr{\pi}}{t_1\app t_2 = t_1\app t_2'}
}{
  \typeof{\Sigma}{\Phi}{\pi_1}{t_2 = t_2'}
}
\\
\infer[\mbox{\typcongeql}]{
  \typeof{\Sigma}{\Phi}{\congeql{\pi}}{(t_1 = t_2) = (t_1' = t_2)}
}{
  \typeof{\Sigma}{\Phi}{\pi}{t_1 = t_1'}
}
\qquad
\infer[\mbox{\typcongeqr}]{
  \typeof{\Sigma}{\Phi}{\congeqr{\pi}}{(t_1 = t_2') = (t_1 = t_2)}
}{
  \typeof{\Sigma}{\Phi}{\pi}{t_2 = t_2'}
}
\\
\infer[\mbox{\typbeta}]{
  \typeof{\Sigma}{\Phi}{\tbetaconv{x}{\tau}{t_1}{t_2}}{(\tabs{x}{\tau}{t_1})\app t_2 = t_2[t_1/x]}
}{
  \typeof{\Sigma}{\Phi}{\tabs{x}{\tau}{t_1}}{\tau_1 \to \tau_2}
  &
  \typeof{\Sigma}{\Phi}{t_2}{\tau_1}
}
\\
\infer[\mbox{\typconv}]{
  \typeof{\Sigma}{\Phi}{\conv{\pi_1}{\pi_2}}{t_2}
}{
  \typeof{\Sigma}{\Phi}{\pi_1}{t_1}
  &
  \typeof{\Sigma}{\Phi}{\pi_2}{t_1 = t_2}
}
\end{gather*}  
  
\caption{The typing rules of \STLCE}
\label{fig:stlce-typing-rules}
\end{figure}


\section{\STLCC: the contextual simply typed $\lambda$-calculus}
%
In this section we describe \STLCC, a further extension of \STLCE for
internalizing the typing derivations of \STLCE as first-class objects
for the manipulation of open \STLCE terms.

To manipulate open \STLCE terms as data objects, we
internalize the derivations of \STLCE terms as \emph{contextual terms}
of the form $\ct{\Phi}{t}$; it denotes a term $t$ that is closed
w.r.t. and well-typed in the context $\Phi$.
%
To capture derivations in their general forms, we allow the context
$\Phi$ to be parameterized by other contexts. We introduce the
meta-variable $\phi$ to denote a context parameter and give the new
syntax of contexts as follows:
%
\begin{tabbing}
\qquad\=$\Phi$\quad\=$:=$\quad\=\kill
\>$\Phi$  \>$:=$ \>$\bullet \sep \phi \sep \Phi, x:\tau$
\end{tabbing}
%
Note that we allow only the prefix of a context can be parameterized
by a meta-variable. More general forms of parameterization were
presented by Stampoulis and Shao\cite{stampoulis12}. For our
discussion we focus on this simpler form.

We need to talk about conversion of terms between closed contexts,
which we characterized as \emph{simultaneous substitutions} and use
$\sigma$ to denote them. Their syntax is given as follows:
%
\begin{tabbing}
\qquad\=$\sigma$\quad\=$:=$\quad\=\kill
\>$\sigma$  \>$:=$ \>$\bullet \sep \idsub{\phi} \sep \sigma, t$
\end{tabbing}
%
Here $\idsub{\phi}$ denotes an identity substitution for the contexts
represented by the meta-variable $\phi$.

We extend \STLCE terms with meta-variables for contextual terms. They
are denoted by $\metavar{X}{\sigma}$ where $\sigma$ is a simultaneous
substitution to be applied to the contextual term substituted for $X$.

The syntax of \STLCC is described as follows where the newly
introduced syntax is again marked red:
%
\begin{tabbing}
\qquad\=$\tau$\quad\=$:=$\quad\=\kill
\>$\tau$  \>$:=$  \>$\prop \sep a \sep \tau_1 \to \tau_2$\\
\>$t$     \>$:=$  \>$c \sep x \sep \tabs{x}{\tau} t \sep t_1\app t_2 \sep t_1 = t_2 \sep \emphred{X/\sigma}$\\
\>$\pi$   \>$:=$  \>$c_\pi \sep \refl{t} \sep \trans{\pi_1}{\pi_2} \sep \symm{\pi}$\\
\>\>\>$\sep \congtabs{x}{\tau}{\pi} \sep \congappl{\pi} \sep \congappr{\pi} \sep \congeql{\pi} \sep \congeqr{\pi}$\\
\>\>\>$\sep \tbetaconv{x}{\tau}{t_1}{t_2} \sep \conv{\pi_1}{\pi_2}$\\
\>$\emphred{\Up}$  \>$:=$  \>$\emphred{\ct{\Phi}{\tau}}$\\
\>$\emphred{T}$     \>$:=$  \>$\emphred{\ct{\Phi}{t}}$\\
\>$\emphred{\Pi}$   \>$:=$  \>$\emphred{\ct{\Phi}{\pi}}$
\end{tabbing}
%
Here $\Up$, $T$ and $\Pi$ denote contextual types, terms and proof terms, respectively.

To describe typing of terms containing meta-variables, we introduce
contexts for these variables. A meta-variable context $\Phi$ as the
following syntax:
%
\begin{tabbing}
\qquad\=$\Psi$\quad\=$:=$\quad\=\kill
\>$\Psi$  \>$:=$ \>$\bullet \sep \Psi, X:\Up$
\end{tabbing}

The typing rules for \STLCC are then given in
Figure~\ref{fig:stlcc-typing-rules}.
%
The rules for deriving $\ctypeof{\Sigma}{\Psi}{\Phi}{t}{\tau}$ and
$\ctypeof{\Sigma}{\Psi}{\Phi}{\pi}{t}$ are mostly adapted from those
in Figure~\ref{fig:stlce-typing-rules} by adding the meta-context
$\Psi$; we omit these adapted rules. The only new rule \typmetavar is
for typing meta-variables.

%
\begin{figure}[ht!]
\fbox{$\ctypeof{\Sigma}{\Psi}{\Phi}{\sigma}{\Phi'}$}
\\
\begin{gather*}
\infer[]{
  \ctypeof{\Sigma}{\Psi}{\Phi}{\bullet}{\bullet}
}{
}
\qquad
\infer[]{
  \ctypeof{\Sigma}{\Psi}{\Phi}{\idsub{\phi}}{\phi}
}{
  \phi \in \Phi
}
\qquad
\infer[]{
  \ctypeof{\Sigma}{\Psi}{\Phi}{\sigma,t}{\Phi',x:\tau}
}{
  \ctypeof{\Sigma}{\Psi}{\Phi}{t}{\tau}
  &
  \ctypeof{\Sigma}{\Psi}{\Phi}{\sigma}{\Phi'}
}
\end{gather*}
\\
\fbox{$\ctypeof{\Sigma}{\Psi}{\Phi}{t}{\tau}$} 
\\
\begin{gather*}
  \ldots \mbox{(adapted from Figure~\ref{fig:stlce-typing-rules} by adding $\Psi$)}
  \\
  \\
  \infer[\mbox{\typmetavar}]{
    \ctypeof{\Sigma}{\Psi}{\Phi}{\metavar{X}{\sigma}}{\tau}
  }{
    X:\ct{\Phi'}{\tau} \in \Psi
    & 
    \ctypeof{\Sigma}{\Psi}{\Phi}{\sigma}{\Phi'}
  }
\end{gather*}
\\
\fbox{$\ctypeof{\Sigma}{\Psi}{\Phi}{\pi}{t}$} 
\\
\begin{gather*}
  \ldots \mbox{(adapted from Figure~\ref{fig:stlce-typing-rules} by adding $\Psi$)}
\end{gather*}
\\
\fbox{$\ctypeof{\Sigma}{\Psi}{\Phi}{T}{\Up}$}
\begin{gather*}
  \infer[\mbox{\typctxterm}]{
    \cmtypeof{\Sigma}{\Psi}{\ct{\Phi}{t}}{\ct{\Phi}{\tau}}
  }{
    \ctypeof{\Sigma}{\Psi}{\Phi}{t}{\tau}
  }
\end{gather*}
\\
\fbox{$\ctypeof{\Sigma}{\Psi}{\Phi}{\Pi}{T}$}
\begin{gather*}
  \infer[\mbox{\typctxpfterm}]{
    \cmtypeof{\Sigma}{\Psi}{\ct{\Phi}{\pi}}{\ct{\Phi}{t}}
  }{
    \ctypeof{\Sigma}{\Psi}{\Phi}{\pi}{t}
  }
\end{gather*}

\caption{The typing rules of \STLCC}
\label{fig:stlcc-typing-rules}
\end{figure}


\section{Algorithmic equality of \STLCC}

In this section, we define alternative typing rules of \STLCC that
check equality of terms algorithmically and that do not generate proof
terms. We prove that the two type system are equivalent in the next
section.

There are two choices:

\begin{itemize}
\item We can follow the normalization \& comparison approach;
\item We can follow the type directed checking approach in the
  literature~\cite{harper05,cave18}\cite[Chapter~6]{pierce04}.
\end{itemize}

\section{Equivalence of declarative and algorithmic equality of \STLCC}

Depending on the choice of algorithmic equality,
%
\begin{itemize}
\item we prove normalization and confluence of the evaluation in
  \STLCC, from which we derive the equivalence; or
\item we define a kripke logical relation and prove it implies
  algorithmic equality and it is implied by declarative
  equality~\cite{harper05,cave18}\cite[Chapter~6]{pierce04}.
\end{itemize}

%% Bibliography
\bibliography{refs} \bibliographystyle{abbrv}

\end{document}
